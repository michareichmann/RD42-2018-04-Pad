%%%%%%%%%%%%%%%%%%%%%%%% FRAME 0 %%%%%%%%%%%%%%%%%%%%%%%%%%%%%%%
\begin{frame}{Pedestal Distributions at \SI{\sim10}{\mega\hertz\per\centi\meter^2}}
 
	\vspace*{-15pt}\subfigspc{PD0}{PD1}{.5}{scCVD with \SI{6}{\deci\bel} attenuation}{pCVD}\vspace*{-10pt}

	\begin{itemize} \itemfill
		\item applying the same cuts as for signal
		\item distribution agrees well with Gaussian even at high rates
		\item extract noise by taking the sigma of the Gaussian fit
		\item noise similar for scCVD and pCVD diamond
	\end{itemize}
 
\end{frame}
%%%%%%%%%%%%%%%%%%%%%%%% FRAME 1 %%%%%%%%%%%%%%%%%%%%%%%%%%%%%%%
\begin{frame}{Signal Distributions at \SI{\sim10}{\mega\hertz\per\centi\meter^2}}
 
	\vspace*{-15pt}\subfigspc{SD0}{SD1}{.5}{scCVD with \SI{6}{\deci\bel} attenuation}{pCVD}\vspace*{-10pt}

	\begin{itemize} \itemfill
		\item event based correction by the mean of the noise (baseline offset)
		\item pCVD signal smaller and smeared by different regions in the diamond
		\item FWHM/MPV:
		\begin{itemize}\vspace*{3pt}
			\item scCVD: \SI{\sim.3}{}
			\item pCVD: \SI{\sim1}{}
		\end{itemize}

	\end{itemize}
 
\end{frame}
%%%%%%%%%%%%%%%%%%%%%%%% FRAME 2 %%%%%%%%%%%%%%%%%%%%%%%%%%%%%%%
\begin{frame}{Signal Maps}
 
	\begin{figure}\vspace*{-15pt}
		\centering
		\subfigp{SM0}{.5}{scCVD with \SI{6}{\deci\bel} attenuation}
		\subfigp{SM1}{.5}{pCVD}
	\end{figure}
	
	\begin{itemize} \itemfill
		\item flat signal distribution in scCVD
		\item signal response depending on region in the pCVD
	\end{itemize}
 
\end{frame}

%%%%%%%%%%%%%%%%%%%%%%%% FRAME 2.1 %%%%%%%%%%%%%%%%%%%%%%%%%%%%%
\begin{frame}{Signal Regions}
 
	\vspace*{-15pt}\subfigsp{SigMapB2H}{SigMapB2L}{.35}
	\vspace*{-15pt}\subfigsp{DistB2H}{DistB2L}{.35}
	
	\begin{itemize} \itemfill
		\item Landau gets narrower for high region (FWHM/MPV: \SI{\sim1.5}{})
		\item stays similar for low region (FWHM/MPV: \SI{\sim.85}{})
	\end{itemize}
 
\end{frame}
%%%%%%%%%%%%%%%%%%%%%%%% FRAME 3 %%%%%%%%%%%%%%%%%%%%%%%%%%%%%%%
\begin{frame}{Rate Studies}

	
	\vspace*{-15pt}
	\figp{B2Oct151}{.65} 
	
	\begin{itemize}
		\item systematically checking several up and down scans
		\item pumping required in the beginning to reach stable pulse height
		\item random scans to rule out systematic effects of the up/down scan
	\end{itemize}
	
\end{frame}
%%%%%%%%%%%%%%%%%%%%%%%% FRAME 4 %%%%%%%%%%%%%%%%%%%%%%%%%%%%%%%
\begin{frame}{Rate Studies in Non-Irradiated scCVD}

	
	\only<1>{\vspace*{-15pt}\figp{S129Scans1}{.65}}
	\only<2>{\vspace*{-15pt}\vspace*{.1\textheight}\figp{S129P}{.45}\vspace*{.1\textheight}}
	
	\begin{itemize}\itemfill
		\item lowest rate point scaled to 1
		\item scCVD diamond shows now rate dependence within the measurement precision
		\item noise stays the same
	\end{itemize}
	
\end{frame}
%%%%%%%%%%%%%%%%%%%%%%%% FRAME 4.1 %%%%%%%%%%%%%%%%%%%%%%%%%%%%%%%
\begin{frame}{pCVD - Unirradiated Negative}
 
	
	\vspace*{-20pt}\figp{Unirrad}{.7}\vspace*{-10pt}
	
	\begin{itemize} \itemfill
		\item all unirradiated poly have slight rate dependence with similar behaviour
		\item first up/down scan usually with lower pulse height \ra pumping
		\item \ra excluding first first up/down scan
	\end{itemize}
 
\end{frame}
%%%%%%%%%%%%%%%%%%%%%%%% FRAME 4.2 %%%%%%%%%%%%%%%%%%%%%%%%%%%%%%
\begin{frame}{pCVD - Unirradiated Positive}
 
	
	\vspace*{-20pt}\figp{UnirradPos}{.7}\vspace*{-10pt}
	
	\begin{itemize} \itemfill
		\item II6-97 does not hold higher voltage
		\item longer pumping at lower voltage?
		\item all other diamonds behave very similar for both positive and negative bias
	\end{itemize}
 
\end{frame}
%%%%%%%%%%%%%%%%%%%%%%%% FRAME 5 %%%%%%%%%%%%%%%%%%%%%%%%%%%%%%%
\begin{frame}{Rate Studies in Irradiated pCVD}

	\vspace*{-15pt}
	\figp{B2Scans1}{.65}
	
	\begin{itemize}\itemfill
		\item lowest rate point scaled to 1
		\item pulse height very stable after irradiation
		\item noise stays the same
% 		\item signal degradation due to radiation damage (no absolute calibration)
	\end{itemize}

\end{frame}